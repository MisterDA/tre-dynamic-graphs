\documentclass[11pt,a4paper]{article}
\usepackage{amsmath,amssymb}
\usepackage[top=2cm,bottom=2cm,left=2cm,right=2cm]{geometry}

\title{Dynamic Graphs}

\begin{document}

\section{Notions}

\paragraph{Transports Publics}
L'idée est de représenter un réseau de transport public (métros,
trams, bus, …) par un graphe, et d'étudier les modèles et les
algorithmes pour une panoplie de problèmes~:
\begin{itemize}
\item Recherche du chemin le plus rapide
\item Recherche du chemin avec la date d'arrivée la plus précoce
\item Multi-critères~: avec le moins de changements
\item Rechercher tous les chemins dans une plage horaire
\end{itemize}


\paragraph{Graphes Dynamiques}
Les modèles varient autour de l'idée de \textit{graphes
  dynamiques}. Les noeuds représentent des stations, et les arrêtes
les lignes entre les stations.

\begin{itemize}
\item Un graphe dynamique peut être vu comme une séquence de graphes
  au cours du temps \(\mathcal{N} = \dots, \mathcal{N}_{t-1},
  \mathcal{N}_{t}, \mathcal{N}_{t+1}, \dots\). La latence d'un arc ou
  d'un noeud est une simple étiquette du noeud~\cite{xuan2003computing}.
\item Une autre présentation~\cite{casteigts2012time}~:
  \(\mathcal{G} = (V, E, \mathcal{T}, \rho, \zeta, \psi, \varphi)\),
  où~:
  \begin{itemize}
  \item \(\rho : E \times \mathcal{T} \to \mathbb{B}\), fonction de
    présence des arrêtes,
  \item \(\zeta : E \times \mathcal{T} \to \mathbb{T}\), fonction de
    latence des arrêtes,
  \item \(\psi : V \times \mathcal{T} \to \mathbb{B}\), fonction de
    présence des noeuds,
  \item \(\varphi : V \times \mathcal{T} \to \mathbb{T}\), fonction de
    latence des noeuds
  \end{itemize}
\end{itemize}

\paragraph{Points de vue}
Dans~\cite{casteigts2012time}, on parle de point de vue de
l'évolution. Du point de vue de l'arrête, sa disponibilité et sa
latence change au cours du temps. Pour un noeud, c'est son voisinage.
Pour tout le graphe, c'est un événement. Chaque événement définit un
nouveau sous-graphe statique. Le graphe dynamique est une séquence de
graphes statiques tels que \(G_i \neq G_{i+1}\).

\paragraph{Sous-graphes}
On définit assez naturellement la notion de sous-graphe et de
\textit{sous-graphe temporel} dans~\cite{casteigts2012time}
et~\cite{latapy2017stream}.

\paragraph{Voyage}
Un voyage est un chemin dans le graphe respectant les contraintes
temporelles.

\paragraph{Distance}
Deux notions de distance (sur un voyage)~:
\begin{itemize}
\item le voyage le plus court (\textit{hop-count})
\item le voyage le plus rapide (qui prends le moins de temps, ou qui
  termine le premier) (\textbf{toujours pas clair})
\end{itemize}

\paragraph{Algorithmes}
Est introduite en~\cite{xuan2003computing} deux notions de graphes
dynamiques, les \textit{evolving graphs (EG)} et les \textit{timed
  evolving graphs (TEG)}, la différence étant que les \textit{TEG}
disposent d'une fonction de latence sur les arrêtes.

Pour appliquer l'algorithme de Dijkstra, il faut remarquer que dans un
graphe normal, le préfixe d'un plus court chemin est aussi un plus
court chemin. Or, cette propriété n'est pas vérifiée dans les graphes
dynamiques. Il faut donc trouver à chaque fois un chemin préfixe qui
respecte cette propriété.

On peut alors calculer les plus courts (\textit{hop-count}) chemins,
et (plus intéressant) les chemins les plus rapides pour toutes les
paires de chemins.

\section{Papiers}

\paragraph{Time-varying graphs and dynamic networks}
Introduction du concept de TVG.\@Classification des TVG en fonction
des propriétés de connexion du graphe. Etude des phénomènes tels que
le \textit{petit monde}, \textit{fairness \& balance}. Graphes
dynamiques où la fonction de présence suit une loi de probabilité.

\paragraph{Stream graphs and link streams for the modeling of
  interactions over time}
Gros papier qui étend toutes les relations et les sous-structures sur
les graphes aux graphes dynamiques.

\bibliographystyle{plain}
\bibliography{tre}

\end{document}
