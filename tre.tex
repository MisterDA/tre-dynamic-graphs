\documentclass[12pt,a4paper]{article}
\usepackage{polyglossia}
\setmainlanguage{french}
\usepackage{csquotes}
\usepackage{amsmath,amssymb,amsthm}
\usepackage[margin=2cm]{geometry}
\usepackage{tikz}
\usetikzlibrary{positioning}
\usepackage{subfig}
\usepackage[backend=biber, style=alphabetic]{biblatex}
\bibliography{tre.bib}
\usepackage[colorlinks, citecolor=blue]{hyperref}

\title{Graphes Temporels ou Dynamiques}
\author{Antonin Décimo\\
Encadrants~: Michel Habib, Laurent Viennot}

\begin{document}
\maketitle
\section{Introduction}
Un graphe dynamique est un graphe qui évolue au cours du temps,
c'est-à-dire que des nœuds ou des arêtes peuvent apparaître ou
disparaître au cours du temps. On peut s'en servir pour modéliser des
réseaux informatiques, comme du calcul distribué ou de la téléphonie
mobile, ou bien encore des réseaux de transports routiers ou publics.

Plusieurs modèles existent dans la littérature, un des plus simples
étant de considérer un ensemble de nœuds fixé, et seulement des arêtes
qui évoluent entre ces nœuds, ce qui amène à considérer une suite de
graphes statiques.

\begin{figure}[h]
  \centering
  \subfloat[\(G_1\)]{
    \begin{tikzpicture}[node/.style={circle, draw, minimum size=1cm]},
      scale=0.5, transform shape]
      \node[node] (e) {e};
      \node[node, above left=of e] (a) {a};
      \node[node, above right=of e] (b) {b};
      \node[node, below right=of e] (c) {c};
      \node[node, below left=of e] (d) {d};

      \path (a) edge (b);
      \path (e) edge (c);
    \end{tikzpicture}}
  \qquad
  \subfloat[\(G_2\)]{
    \begin{tikzpicture}[node/.style={circle, draw, minimum size=1cm]},
      scale=0.5, transform shape]
      \node[node] (e) {e};
      \node[node, above left=of e] (a) {a};
      \node[node, above right=of e] (b) {b};
      \node[node, below right=of e] (c) {c};
      \node[node, below left=of e] (d) {d};

      \path (a) edge (b) edge [bend left=30] (c);
    \end{tikzpicture}}
  \qquad
  \subfloat[\(G_3\)]{
    \begin{tikzpicture}[node/.style={circle, draw, minimum size=1cm]},
      scale=0.5, transform shape]
      \node[node] (e) {e};
      \node[node, above left=of e] (a) {a};
      \node[node, above right=of e] (b) {b};
      \node[node, below right=of e] (c) {c};
      \node[node, below left=of e] (d) {d};

      \path (c) edge (d) edge (e);
      \path (a) edge (e);
    \end{tikzpicture}}
  \qquad
  \subfloat[\(G_4\)]{
    \begin{tikzpicture}[node/.style={circle, draw, minimum size=1cm]},
      scale=0.5, transform shape]
      \node[node] (e) {e};
      \node[node, above left=of e] (a) {a};
      \node[node, above right=of e] (b) {b};
      \node[node, below right=of e] (c) {c};
      \node[node, below left=of e] (d) {d};

      \path (b) edge (e) edge (c);
      \path (c) edge (d);
    \end{tikzpicture}}
  \caption{Graphe dynamique \(G = (G_1, G_2, G_3, G_4)\)}
\end{figure}

\subsection{Rappels}
Rappelons d'abord quelques notions sur les graphes, qui nous seront
utiles par la suite.

\begin{description}
\item[Graphe \(k\)-régulier] (\textit{\(k\)-regular}) Un graphe est
  \(k\)-régulier si chacun de ses sommets est de degré \(k\).
\item[\(k\)-sommet-connexe] (\textit{\(k\)-vertex-connected}) Un
  graphe est \(k\)-sommet-connexe s'il possède plus de \(k\) sommets
  et s'il reste connexe après en avoir ôté moins de \(k\).
\item[Graphe expanseur] (\textit{expander graph}) Le taux d'expansion
  d'un graphe est une mesure de sa connectivité. Soit \(G = (V, E)\)
  un graphe, et \(W\) un sous-graphe de \(G\). La \textbf{frontière
    extérieure des sommets} (\textit{outer vertex boundary}) est
  l'ensemble des sommets de \(V \setminus W\)ayant au moins un sommet
  dans \(W\).
  \[\partial_{out}(W) = \{v \in V \setminus W | \exists w \in W (w, v)
    \in E\}\]
  On définit alors le \textbf{taux d'expansion des sommets}.
  \[h_{out}(G) = \min_{0 < |W| \leq n/2}\frac{|\partial_{out}(W)|}{|W|}\]
  \(G\) est un graphe \(c\)-expanseur si pour tout \(W \subset V\) tel
  que \(|W| \leq n / 2\), alors \(|\partial(W)| \geq c |W|\).
\end{description}

\section{Modèles de graphes dynamiques}

Les graphes dynamiques (\textit{dynamic graphs}) sont aussi appelés
graphes temporels (\textit{temporal graphs}), graphes de flux
(\textit{stream graphs}), réseaux dynamiques (\textit{dynamic
  network}), graphes évolutifs (\(\textit{evolving graphs}\)),
graphes variants avec le temps (\(\textit{time-varying graphs}\)).

\subsection{Modèles sans latence}

\paragraph{Graphe de flux}\cite{latapy2017stream} (\textit{stream
  graph}) Un graphe de flux \(S = (T, V, W, E)\) est un ensemble fini
de nœuds \(V\), un ensemble d'instants \(T\), un ensemble de nœuds
temporels \(W \subseteq T \times V\) (un ensemble de nœuds indexés par
un temps), et un ensemble d'arêtes
\(E \subseteq T \times (V \otimes V)\) tel que si \((t, uv) \in E\)
alors \((t, u) \in W\) et \((t, v) \in W\) (si une arête existe à un
temps \(t\), alors les deux nœuds de l'arête existent aussi à ce
temps).

\paragraph{Flux de liens} (\textit{link stream}) Si tous les nœuds du
graphe de flux sont présents à chaque instant, on parle de flux de
liens.

\paragraph{Graphe évoluant}\cite{kuhn2011dynamic} (\textit{evolving
  graph}) Soit un graphe dynamique \(G = (V, E)\), où \(V\) est un
ensemble de nœuds, et
\(E : \mathbb{N}^{+} \to \mathcal{P}(V \times V)\) est une fonction
attribuant à chaque tour \(r \in \mathbb{N}^{+}\) un ensemble d'arêtes
\(E(r)\) pour ce tour. Ici l'ensemble des nœuds ne varie pas.

\subsection{Modèles avec latence}

\paragraph{Graphe variant avec le temps}\cite{casteigts2012time} La
principale différence avec les modèles ci-dessus est que la présence
des entités n'est plus définie par un instant, mais par un intervalle
de temps. On note \(\mathbb{T}\) le domaine temporel. Les auteurs
proposent de rajouter un label \(L\) sur chaque arête, et préfèrent
indiquer la présence d'arêtes et de nœuds par des fonctions. On a donc
\(G = (V, E \subseteq L \times V \otimes V, T \subseteq \mathbb{T},
\rho, \zeta)\), où~:
\begin{itemize}
\item \(\rho : E \times T \to \{\top, \bot\}\) est appelée fonction
  de présence et indique si une arête donnée est disponible à un temps
  donné.
\item \(\zeta : E \times T \to \mathbb{T}\) est appelée fonction de
  latence et indique le temps de parcours d'une arête donnée pour un
  départ à un temps donné (la latence d'une arête pouvant changer au
  cours du temps).
\end{itemize}

On peut à l'instar des arêtes étendre ce modèle aux nœuds du graphe,
avec \(\psi\) la fonction de présence et \(\varphi\) la fonction de
latence. En choisissant une fonction de latence constante, on se
ramène à un modèle sans latence.

\section{Notions et Propriétés}

L'article~\cite{kuhn2010distributed} présente deux notions intéressantes.

\paragraph{Connexité \(T\)‑intervalle}
Un graphe dynamique \(G = (V, E)\) est \(T\)‑intervalle connexe
(\textit{\(T\)‑interval connected}) pour \(T \geq 1\) si pour tout
\(r \in \mathbb{N}\), le graphe statique
\(G_{r,T} := (V, \bigcap_{i=r}^{r+T-1} E(r))\) est connexe. Le graphe
est \(\infty\)‑intervalle connexe s'il existe un graphe statique
connexe \(G' = (V, E')\) tel que pour tous
\(r \in \mathbb{N}, E' \subseteq E(r)\)

\paragraph{Causalité de Lamport}
Soit un graphe dynamique \(G = (V, E, T)\), on peut définir un ordre
\(\rightarrow\) sur \((V, T)\), tel que
\((u, r) \rightarrow (v, r')\) ssi \((r' = r + 1)\) et
\(\{u, v\} \in E(r)\). L'\textit{ordre causal} \(\rightsquigarrow\)
est défini comme clôture réflexive et transitive de \(\rightarrow\).

\subsection{Voyages et distances}

Les articles~\cite{xuan2003computing},~\cite{casteigts2012time}, et
~\cite{latapy2017stream} définissent les concepts de chemins,
distances, et voyages dans les graphes dynamiques. Nous allons
considérer le modèle de~\cite{casteigts2012time}
(\(G = (V, E, T, \rho, \zeta)\)), plus complet.

\paragraph{Voyages, chemins} (\textit{journeys, paths}) Un voyage est
une séquence \(J = \{(e_1, t_1), \dots, (e_k, t_k)\}\) telle que
\({e_1, \dots, e_k}\) est une marche dans le graphe,
\(\rho(e_i, t_i) = \top\), et \(t_{i+1} \geq t_i + \zeta(e_i, t_i)\)
pour tout \(i < k\). Selon l'application, on pourra supposer
\(\rho_{[t_i, t_i + \zeta(e_i, t_i)]}(e_i) = \top\), c'est-à-dire que
l'arête reste présente tandis qu'elle est traversée.

Note~: les auteurs ne proposent pas de définition tenant compte de la
présence ni de la latence des nœuds.

Dans les graphes dynamiques, les chemins sont asymétriques.

\paragraph{Distance topologique} La distance topologique entre deux
nœuds dans un graphe est la longueur minimum (en nombre d'arêtes) des
chemins entre ces nœuds. On parle de \textbf{chemin le plus court}
(\textit{shortest path}).

\paragraph{Distance temporelle}
On dit qu'un chemin est \textbf{le plus rapide} (\textit{fastest
  path}) s'il est de durée (la somme des latences des nœuds et des
arêtes rencontrés) minimum.\\
Le temps pour atteindre \((t, v)\) depuis \(u\) au temps \(\alpha\)
est \(\mathcal{T}_{\alpha}(u, (t, v)) = \omega - \alpha\), où
\(\omega\) est la plus petite valeur pour laquelle il existe un chemin
de \((\alpha, u)\) à \((\omega, v)\). Un tel chemin est appelé
\textbf{premier chemin} (\textit{foremost path}).

\subsection{Structures}

En généralisant le langage habituel des graphes, on peut définir les
\textbf{sous-graphes dynamiques}, soit en considérant les sous-graphes
induits par des sous-ensembles de nœuds ou d'arêtes, soit en se
restreignant à un intervalle temporel.

\section{Algorithmes}

\subsection{Exercice~: algorithme du flot maximum}
\paragraph{Flot} On considère un graphe orienté \(G = (V, E)\) et deux
sommets \(s, t \in V\). Les arcs sont munis de capacités \(c : E \to
\mathbb{R}^{+}\).
Un flot sur \(G\) est un vecteur indexé par \(E\) qui vérifie (lois de
Kirchhoff)~:
\begin{itemize}
\item \(\forall (u, v) \in E, 0 \leq \phi(u, v) \leq c(u, v)\)
\item \(\forall v \in V, \sum_{(u, v) \in E} \phi(u, v) = \sum_{(v, w)
    \in E} \phi(v, w)\)
\end{itemize}
On cherche un flot qui maximise \(\phi(t, s)\).


Quels sont les problèmes algorithmiques que l'on peut rencontrer dans
ce cadre~?

\begin{description}
\item[Diamètre dynamique] (\textit{dynamic diameter}, \textit{flooding
    time}) Soit \(G = (V, E)\) un graphe dynamique et \(u \in V\). Si
  \(\forall v \in V, (u, t) \rightsquigarrow (v, t' + D)\), alors
  \(D\) est le diamètre dynamique du graphe. Il faut comprendre que si
  \(u\) envoie un message au temps \(t\), alors le message mettra au
  maximum \(D\) tours pour atteindre tous les nœuds du graphe.
\item[Couverture] (\textit{cover time})
\item[Plus court chemin] L'algorithme de Dijkstra fonctionne dans les
  graphes standards en remarquant que le préfixe d'un plus court
  chemin est aussi un plus court chemin. Or, cette propriété n'est
  pas vérifiée dans les graphes dynamiques.\\
  Implémenter et présenter Dijkstra~\cite{xuan2003computing} sur les
  graphes dynamiques.
\end{description}

\printbibliography{}

\end{document}

%%% Local Variables:
%%% mode: latex
%%% TeX-master: t
%%% End:
