\documentclass[11pt,a4paper]{article}
\usepackage{amsmath,amssymb}
\usepackage[top=2cm,bottom=2cm,left=2cm,right=2cm]{geometry}

\title{Dynamic Graphs}
\author{Antonin Décimo}

\begin{document}

\paragraph{Transports Publics}
L'idée est de représenter un réseau de transport public (métros,
trams, bus, …) par un graphe, et d'étudier les modèles et les
algorithmes pour une panoplie de problèmes~:
\begin{itemize}
\item Recherche du chemin le plus rapide
\item Recherche du chemin avec la date d'arrivée la plus précoce
\item Multi-critères~: avec le moins de changements
\item Rechercher tous les chemins dans une plage horaire
\end{itemize}


\paragraph{Graphes Dynamiques}
Les modèles varient autour de l'idée de \textit{graphes
  dynamiques}. Les noeuds représentent des stations, et les arrêtes
les lignes entre les stations.

\begin{itemize}
\item Un graphe dynamique peut être vu comme une séquence de graphes
  au cours du temps \(\mathcal{N} = \dots, \mathcal{N}_{t-1},
  \mathcal{N}_{t}, \mathcal{N}_{t+1}, \dots\). La latence d'un arc ou
  d'un noeud est une simple étiquette du noeud~\cite{xuan2003computing}.
\item Dans~\cite{casteigts2012time}, le formalisme est plus élégant (à
  mon sens).
  \(\mathcal{G} = (V, E, \mathcal{T}, \rho, \zeta, \psi, \varphi)\),
  où~:
  \begin{itemize}
  \item \(\rho : E \times \mathcal{T} \to \mathbb{B}\), fonction de
    présence des arrêtes,
  \item \(\zeta : E \times \mathcal{T} \to \mathbb{T}\), fonction de
    latence des arrêtes,
  \item \(\psi : V \times \mathcal{T} \to \mathbb{B}\), fonction de
    présence des noeuds,
  \item \(\varphi : V \times \mathcal{T} \to \mathbb{T}\), fonction de
    latence des noeuds
  \end{itemize}.
\end{itemize}

\paragraph{Algorithmes}
Est introduite en~\cite{xuan2003computing} deux notions de graphes
dynamiques, les \textit{evolving graphs (EG)} et les \textit{timed
  evolving graphs (TEG)}, la différence étant que les \textit{TEG}
disposent d'une fonction de latence sur les arrêtes.

Pour appliquer l'algorithme de Dijkstra, il faut remarquer que dans un
graphe normal, le préfixe d'un plus court chemin est aussi un plus
court chemin. Or, cette propriété n'est pas vérifiée dans les graphes
dynamiques. Il faut donc trouver à chaque fois un chemin préfixe qui
respecte cette propriété.

On peut alors calculer les plus courts (\textit{hop-count}) chemins,
et (plus intéressant) les chemins les plus rapides pour toutes les
paires de chemins.

\bibliographystyle{plain}
\bibliography{tre}

\end{document}
