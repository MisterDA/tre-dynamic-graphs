\documentclass[12pt,a4paper]{article}
\usepackage{polyglossia}
\setmainlanguage{french}
\usepackage{csquotes}
\usepackage{amsmath,amssymb,amsthm}
\usepackage[margin=2cm]{geometry}
\usepackage{tikz}
\usetikzlibrary{positioning}
\usepackage{subfig}
\usepackage[backend=biber, style=alphabetic]{biblatex}
\bibliography{tre.bib}
\usepackage{hyperxmp}
\usepackage[colorlinks, citecolor=blue]{hyperref}
\usepackage{xcolor}
\usepackage{algorithm2e}
\usepackage[type={CC}, modifier={by-sa}, version={4.0}]{doclicense}

\title{Graphes Temporels ou Dynamiques}
\author{Antonin Décimo\\
Encadrants~: Michel Habib, Laurent Viennot}

\begin{document}
\maketitle
\doclicenseThis{}
\section{Introduction}
Un graphe dynamique est un graphe qui évolue au cours du temps,
c'est-à-dire que des nœuds ou des arêtes peuvent apparaître ou
disparaître au cours du temps. On peut s'en servir pour modéliser des
réseaux informatiques, comme du calcul distribué ou de la téléphonie
mobile, ou bien encore des réseaux de transports routiers ou publics.

Plusieurs modèles existent dans la littérature, un des plus simples
étant de considérer un ensemble de nœuds fixé, et seulement des arêtes
qui évoluent entre ces nœuds, ce qui amène à considérer une suite de
graphes statiques.

\begin{figure}[h]
  \centering
  \subfloat[\(G_1\)]{
    \begin{tikzpicture}[node/.style={circle, draw, minimum size=1cm]},
      scale=0.5, transform shape]
      \node[node] (e) {e};
      \node[node, above left=of e] (a) {a};
      \node[node, above right=of e] (b) {b};
      \node[node, below right=of e] (c) {c};
      \node[node, below left=of e] (d) {d};

      \path (a) edge (b);
      \path (e) edge (c);
    \end{tikzpicture}}
  \qquad
  \subfloat[\(G_2\)]{
    \begin{tikzpicture}[node/.style={circle, draw, minimum size=1cm]},
      scale=0.5, transform shape]
      \node[node] (e) {e};
      \node[node, above left=of e] (a) {a};
      \node[node, above right=of e] (b) {b};
      \node[node, below right=of e] (c) {c};
      \node[node, below left=of e] (d) {d};

      \path (a) edge (b) edge [bend left=30] (c);
    \end{tikzpicture}}
  \qquad
  \subfloat[\(G_3\)]{
    \begin{tikzpicture}[node/.style={circle, draw, minimum size=1cm]},
      scale=0.5, transform shape]
      \node[node] (e) {e};
      \node[node, above left=of e] (a) {a};
      \node[node, above right=of e] (b) {b};
      \node[node, below right=of e] (c) {c};
      \node[node, below left=of e] (d) {d};

      \path (c) edge (d) edge (e);
      \path (a) edge (e);
    \end{tikzpicture}}
  \qquad
  \subfloat[\(G_4\)]{
    \begin{tikzpicture}[node/.style={circle, draw, minimum size=1cm]},
      scale=0.5, transform shape]
      \node[node] (e) {e};
      \node[node, above left=of e] (a) {a};
      \node[node, above right=of e] (b) {b};
      \node[node, below right=of e] (c) {c};
      \node[node, below left=of e] (d) {d};

      \path (b) edge (e) edge (c);
      \path (c) edge (d);
    \end{tikzpicture}}
  \caption{Graphe dynamique \(G = (G_1, G_2, G_3, G_4)\)}
\end{figure}

\subsection{Rappels}
Rappelons d'abord quelques notions sur les graphes, qui nous seront
utiles par la suite.

\begin{description}
\item[Graphe \(k\)-régulier] (\textit{\(k\)-regular}) Un graphe est
  \(k\)-régulier si chacun de ses sommets est de degré \(k\).
\item[\(k\)-sommet-connexe] (\textit{\(k\)-vertex-connected}) Un
  graphe est \(k\)-sommet-connexe s'il possède plus de \(k\) sommets
  et s'il reste connexe après en avoir ôté moins de \(k\).
\item[Graphe expanseur] (\textit{expander graph}) Le taux d'expansion
  d'un graphe est une mesure de sa connectivité. Soit \(G = (V, E)\)
  un graphe, et \(W\) un sous-graphe de \(G\). La \textbf{frontière
    extérieure des sommets} (\textit{outer vertex boundary}) est
  l'ensemble des sommets de \(V \setminus W\) ayant au moins un voisin
  dans \(W\).
  \[\partial(W) = \left\{v \in V \setminus W \,\middle|\, \exists w
      \in W, (w, v) \in E\right\}\]

  On définit alors le \textbf{taux d'expansion des sommets}.
  \[h(G) = \min_{0 < |W| \leq n/2}\frac{|\partial(W)|}{|W|}\]
  \(G\) est un graphe \(c\)-expanseur si pour tout \(W \subset V\) tel
  que \(|W| \leq n / 2\), alors \(|\partial(W)| \geq c |W|\).
\end{description}

\section{Modèles de graphes dynamiques}

Les graphes dynamiques (\textit{dynamic graphs}) sont aussi appelés
graphes temporels (\textit{temporal graphs}), graphes de flux
(\textit{stream graphs}), réseaux dynamiques (\textit{dynamic
  network}), graphes évolutifs (\(\textit{evolving graphs}\)),
graphes variants avec le temps (\(\textit{time-varying graphs}\)).

\subsection{Modèles sans latence}

\paragraph{Graphe de flux}\cite{latapy2017stream} (\textit{stream
  graph}) Un graphe de flux \(S = (T, V, W, E)\) est un ensemble fini
de nœuds \(V\), un ensemble d'instants \(T\), un ensemble de nœuds
temporels \(W \subseteq T \times V\) (un ensemble de nœuds indexés par
un temps), et un ensemble d'arêtes
\(E \subseteq T \times (V \otimes V)\) tel que si \((t, uv) \in E\)
alors \((t, u) \in W\) et \((t, v) \in W\) (si une arête existe à un
temps \(t\), alors les deux nœuds de l'arête existent aussi à ce
temps).

\paragraph{Flux de liens}\cite{latapy2017stream} (\textit{link
stream}) Si tous les nœuds du graphe de flux sont présents à chaque
instant, on parle de flux de liens.

\paragraph{Graphe évoluant}\cite{kuhn2011dynamic} (\textit{evolving
  graph}) Soit un graphe dynamique \(G = (V, E)\), où \(V\) est un
ensemble de nœuds, et
\(E : \mathbb{N}^{+} \to \mathcal{P}(V \times V)\) est une fonction
attribuant à chaque tour \(r \in \mathbb{N}^{+}\) un ensemble d'arêtes
\(E(r)\) pour ce tour. Ici l'ensemble des nœuds ne varie pas.\\


Il n'y a pas de différence fondamentale entre les \textit{link
  streams} et les \textit{evolving graphs}.

\subsection{Modèles avec latence}

\paragraph{Time-Varying Graphs}\cite{casteigts2012time} (TVG) La
principale différence avec les modèles ci-dessus est que la présence
des entités n'est plus définie par un instant, mais par un intervalle
de temps. On note \(\mathbb{T}\) le domaine temporel. Les auteurs
proposent de rajouter un label \(L\) sur chaque arête, et préfèrent
indiquer la présence d'arêtes et de nœuds par des fonctions. On a donc
\(G = (V, E \subseteq L \times V \otimes V, T \subseteq \mathbb{T},
\rho, \zeta)\), où~:
\begin{itemize}
\item \(\rho : E \times T \to \{\top, \bot\}\) est appelée fonction
  de présence et indique si une arête donnée est disponible à un temps
  donné.
\item \(\zeta : E \times T \to \mathbb{T}\) est appelée fonction de
  latence et indique le temps de parcours d'une arête donnée pour un
  départ à un temps donné (la latence d'une arête pouvant changer au
  cours du temps).
\end{itemize}

On peut à l'instar des arêtes étendre ce modèle aux nœuds du graphe,
avec \(\psi\) la fonction de présence et \(\varphi\) la fonction de
latence. En choisissant une fonction de latence constante à 1, on se
ramène à un modèle sans latence.

\section{Notions et Propriétés}

L'article~\cite{kuhn2010distributed} présente deux notions intéressantes.

\paragraph{Connexité \(T\)‑intervalle}
Un graphe dynamique \(G = (V, E)\) est \(T\)‑intervalle connexe
(\textit{\(T\)‑interval connected}) pour \(T \geq 1\) si pour tout
\(r \in \mathbb{N}\), le graphe statique
\(G_{r,T} := (V, \bigcap_{i=r}^{r+T-1} E(r))\) est connexe. Le graphe
est \(\infty\)‑intervalle connexe s'il existe un graphe statique
connexe \(G' = (V, E')\) tel que pour tous
\(r \in \mathbb{N}, E' \subseteq E(r)\)

\paragraph{Causalité de Lamport}
Soit un graphe dynamique \(G = (V, E, T)\), on peut définir un ordre
\(\rightarrow\) sur \((V, T)\), tel que
\((u, r) \rightarrow (v, r')\) ssi \((r' = r + 1)\) et
\(\{u, v\} \in E(r)\). L'\textit{ordre causal} \(\rightsquigarrow\)
est défini comme clôture réflexive et transitive de \(\rightarrow\).

\subsection{Voyages et distances}

Les articles~\cite{xuan2003computing},~\cite{casteigts2012time}, et
~\cite{latapy2017stream} définissent les concepts de chemins,
distances, et voyages dans les graphes dynamiques. Nous allons
considérer le modèle de~\cite{casteigts2012time}
(\(G = (V, E, T, \rho, \zeta)\)), plus complet.

\paragraph{Voyages, chemins} (\textit{journeys, paths}) Un voyage est
une séquence \(J = \{(e_1, t_1), \dots, (e_k, t_k)\}\) telle que
\({e_1, \dots, e_k}\) est une marche dans le graphe,
\(\rho(e_i, t_i) = \top\), et \(t_{i+1} \geq t_i + \zeta(e_i, t_i)\)
pour tout \(i < k\). Selon l'application, on pourra supposer
\(\rho_{[t_i, t_i + \zeta(e_i, t_i)]}(e_i) = \top\), c'est-à-dire que
l'arête reste présente tandis qu'elle est traversée.

Note~: les auteurs ne proposent pas de définition tenant compte de la
présence ni de la latence des nœuds.

Dans les graphes dynamiques, les chemins sont asymétriques.

\paragraph{Distance topologique} La distance topologique entre deux
nœuds dans un graphe est la longueur minimum (en nombre d'arêtes) des
chemins entre ces nœuds. On parle de \textbf{chemin le plus court}
(\textit{shortest path}).

\paragraph{Distance temporelle}
On dit qu'un chemin est \textbf{le plus rapide} (\textit{fastest
  path}) s'il est de durée (la somme des latences des nœuds et des
arêtes rencontrés) minimum.\\
Le temps pour atteindre \((t, v)\) depuis \(u\) au temps \(\alpha\)
est \(\mathcal{T}_{\alpha}(u, (t, v)) = \omega - \alpha\), où
\(\omega\) est la plus petite valeur pour laquelle il existe un chemin
de \((\alpha, u)\) à \((\omega, v)\). Un tel chemin est appelé
\textbf{premier chemin} (\textit{foremost path}).

\subsection{Structures}

En généralisant le langage habituel des graphes, on peut définir les
\textbf{sous-graphes dynamiques}, soit en considérant les sous-graphes
induits par des sous-ensembles de nœuds ou d'arêtes, soit en se
restreignant à un intervalle temporel.

Pour un graphe dynamique \(\mathcal{G} = (V, E, T, \rho, \zeta)\), on
appelle \textbf{graphe sous-jacent} \(G = (V, E)\) le graphe statique
dans lequel on oublie la dimension temporelle, et qui indique
seulement les nœuds en relation à un certain temps dans \(T\).

\subsection{Graphes dynamiques aléatoires}

Dans~\cite{casteigts2012time} et~\cite{kuhn2011dynamic}, les auteurs
construisent des graphes dynamiques aléatoires en posant que la
fonction de présence d'une arête \(\rho : E \times T \to {0, 1}\) suit
une loi de probabilité. On peut imaginer des graphes dynamiques
suivant des chaînes de Markov, où une arête apparaît au tour suivant
avec une probabilité \(p\) et disparaît avec une probabilité \(q\),
indépendamment des autres arêtes. Si \(q = 1 - p\), chaque graphe
statique est un graphe d'Erdős-Rényi.

\section{Algorithmes}

\subsection{Exercice~: algorithme du flot maximum}
\paragraph{Flot} On considère un graphe orienté \(G = (V, E)\) et deux
sommets \(s, t \in V\). Les arcs sont munis de capacités \(c : E \to
\mathbb{R}^{+}\).
Un flot sur \(G\) est un vecteur indexé par \(E\) qui vérifie (lois de
Kirchhoff)~:
\begin{itemize}
\item \(\forall (u, v) \in E, 0 \leq \phi(u, v) \leq c(u, v)\)
\item \(\forall v \in V, \sum_{(u, v) \in E} \phi(u, v) = \sum_{(v, w)
    \in E} \phi(v, w)\)
\end{itemize}
On cherche un flot qui maximise \(\phi(t, s)\).

\begin{figure}[h]
  \centering
  \begin{tikzpicture}[node/.style={circle, draw, minimum size=1cm]},
    scale=0.5, transform shape]
    \node[node] (s) {s};
    \node[node, above right=of s] (a) {};
    \node[node, right=of s] (b) {};
    \node[node, right=of a] (c) {};
    \node[node, right=of b] (d) {};
    \node[node, below right=of c] (t) {t};

    \path [->] (s) edge (a) edge (b);
    \path [->] (a) edge (c) edge (d);
    \path [->] (b) edge (d);
    \path [->] (c) edge (t);
    \path [->] (d) edge (t);
    \path [->] (t) edge [bend left=30, dotted] (s);
  \end{tikzpicture}
  \caption{Flot}
\end{figure}

\subsection{Problèmes algoritmiques}

Quels sont les problèmes algorithmiques que l'on peut rencontrer dans
ce cadre~?

\begin{description}
\item[Diamètre dynamique] (\textit{dynamic diameter}, \textit{flooding
    time}) Soit \(G = (V, E)\) un graphe dynamique et \(u \in V\). Si
  \(\forall v \in V, (u, t) \rightsquigarrow (v, t' + D)\), alors
  \(D\) est le diamètre dynamique du graphe. Il faut comprendre que si
  \(u\) envoie un message au temps \(t\), alors le message mettra au
  maximum \(D\) tours pour atteindre tous les nœuds du graphe.
\item[Couverture] (\textit{cover time})
\end{description}

\paragraph{Plus court chemin}
(\textit{shortest path}). On considère les chemins les plus courts en
nombre de sauts. Pour un nœud \(u\) au temps \(t\), fonction
\(neighbours : V \times t \to \mathcal{P}(V \times T)\) renvoie
\((v, t')\) si \(v\) est un voisin accessible de \(u\) au temps le
plus tôt \(t' \geq t\).

L'algorithme suivant, présenté dans~~\cite{xuan2003computing}, calcule
les plus courts chemins dans un graphe dynamique. Les auteurs font
plusieurs suppositions~:
\begin{itemize}
\item Les nœuds sont toujours présents. Bien que la structure de
  données intègre une fonction de présence des nœuds, cet algorithme
  n'en tient pas compte.
\item Une arête existe tant qu'on peut la traverser.
\item Les deux nœuds d'une arête existent tant qu'elle-même existe.
\end{itemize}

\begin{algorithm}[H]
  \KwIn{Un graphe dynamique \(G\), un nœud \(s \in V_G\)}
  \KwData{L'arbre \(T\) des paires \((u, t) \in V_G \times
    \mathbb{R}^*_+\); un entier \(d\); un tableau \(earliest : V_G \to
    \mathbb{R}^*_+\)}
  \KwResult{L'arbre des plus courts chemins \(T\), un tableau
    \(location : V_G \to T\)}
  \(T = \{(s, 0)\}\)\;
  \(earliest(s) \leftarrow 0\)\;
  \(\forall u \neq s, earliest(u) \leftarrow \infty\)\;
  \(d \leftarrow 1\)\;
  \(location(s) \leftarrow (s, 0)\)\;
  \While{\(\exists w \in V_G\) tel que \(location(w)\) n'est pas définie}{
    \ForEach{\((u, t) \in T\) à la profondeur \(d\)}{
      \ForEach{\((v, t') \in neighbours((u, t))\)}{
        \If{\(location(v)\) n'est pas définie}{\(location(v)
          \leftarrow (v, t')\)}
        \If{\(earliest(v) > t'\)}{
          \(earliest(v) \leftarrow t'\)\;
          \((v, t')\) est enfant de \((u, t)\) dans \(T\)\;
        }
      }
    }
    \(d \leftarrow d + 1\)\;
  }
\end{algorithm}

Une fois que l'arbre a été calculé, retrouver le chemin de \(s\) à
n'importe quel sommet \(u\) est aisé~: la \(location\) du sommet donne
la paire \((u, t)\) correspondante dans l'arbre. On retrouve le chemin
en cherchant le parent de \((u, t)\) dans l'arbre \(T\).

Une implémentation de cet algorithme réalisée en OCaml est disponible
à l'adresse
\url{https://gist.github.com/MisterDA/6dfb9740ce992ffc2890de83e968eb49}. Les
contraintes de l'algorithme ont été conservées. L'implémentation ne
vérifie pas que les données sont bien valides.

\section{Articles de recherche}
\subsection{Time-varying graphs and dynamic
  networks~\cite{casteigts2012time}}
Les auteurs identifient trois champs d'applications possibles des
graphes dynamiques.
\begin{description}
\item[Delay-Tolerant Networks]~\cite{fall2007delay} La caractéristique
  principale de ces réseaux est la possibilité de ne pas être connexe
  à un moment donné. Les chemins généralement disponibles au cours du
  temps, et la diffusion ou le routage sont réalisables grâce au
  mécanisme de \textit{store-carry-forward} (un exemple est celui du
  courrier postal).
\item[Opportunistic-Mobility Networks] Ce sont aussi des DNT, avec la
  différence que les nœuds sont mobiles. Il existe des réseaux de
  taxis et de bus équipés de routeurs.
\item[Real-World Complex Networks] Dans cette catégorie tombent tous
  les autres types de réseaux (réseaux sociaux, transports, …),
  unifiés par l'utilisation d'un même formalisme et de concepts comme
  les chemins, la distance, la connexité, en fonction de l'évolution
  du réseau.
\end{description}

Les auteurs définissent ensuite les \textit{time-varying graphs}, un
modèle de graphe dynamique avec latence. Ils se contente dans le reste
de l'article d'évoquer la latence sur les arêtes. La latence sur un
nœud peut être imaginée comme un temps de calcul, ou de transfert.

Les auteurs décrivent le graphe sous-jacent d'un graphe dynamique,
puis développent la notion de point de vue par rapport à l'évolution
du graphe. Pour une arête, l'évolution se cantonne à une variation de
disponibilité (l'arête est-elle présente ou non) et de latence. Pour
un nœud, il s'agit de changements dans le voisinage. Enfin, par
rapport au graphe, l'évolution est considérée comme un séquence de
graphes statiques \(G_1, G_2, \dots\), où chaque graphe statique est
défini par un événement topologique (et \(G_i \neq G_{i+1}\)), ou bien
par un instantané au temps \(t = i\) (et on peut avoir
\(G_i = G_{i+1}\)). On parle d'\textit{evolving graphs} pour
représenter un graphe dynamique par une séquence de graphes statiques.

Les auteurs rappellent les définitions de sous-graphe d'un graphe
dynamique, de chemin, de distance, et d'autres concepts.

La contribution principale de l'article est la classification des TVG
selon des propriétés. En voici quelques-unes~:

\begin{itemize}
\item \(\exists u \in V : \forall v \in V, u \rightsquigarrow v\). On
  peut diffuser dans tout le graphe à partir d'un nœud.
\item \(\exists u \in V : \forall v \in V, v \rightsquigarrow u\). Il
  existe un nœud qui peut être atteint par tous les autres. C'est une
  condition nécessaire pour un calcul réparti sur tout le réseau, avec
  un nœud de sortie.
\item
  \(\forall e \in E, \forall t \in T, \forall k \in \mathbb{N},
  \rho(e, t) = \rho(e, t + kp)\), pour un \(p \in T\), et \(G\) est
  connexe. Cette propriété de périodicité peut être retrouvée dans les
  réseaux de transports, ou les DNT.\@ On peut s'en servir pour
  réutiliser des calculs de chemins, modulo \(p\).
\end{itemize}

Sont présentés les phénomènes de \textit{small world}, où la distance
(en termes de sauts) entre deux nœuds croît logarithmiquement par
rapport au nombre de nœuds du graphe; et la \textit{fairness and
  balance}, l'écart-type de l'excentricité temporelle des nœuds.\\

En conclusion, les auteurs remarquent que peu de problèmes autour des
graphes dynamiques ont été étudiés, mais si une situation peut être
modélisée par un graphe dynamique, alors on peut se servir du
formalisme et des phénomènes présentés pour l'étudier. Une situation
intéressante est celle où on modélise les interactions d'un groupe
social par un graphe dynamique et de on tente de minimiser
l'éloignement temporel moyen afin d'accélérer les échanges.

\subsection{Dynamic Netwoks: Models and
  Algorithms~\cite{kuhn2011dynamic}}

Il s'agit essentiellement d'une revue de la littérature, mais qui
présente néanmoins deux résultats. Tout d'abord, les auteurs
définissent leur cadre d'étude~: des réseaux dynamiques dans lesquels
les changements sont incessants, incontrôlables. Les algorithmes
doivent terminer.

Le modèle utilisé est celui des \textit{evolving graphs}, le temps est
discrétisé. La fonction de présence des arêtes est dynamique, et
déterminée par un \textbf{modèle d'adversaire}. On a
l'\textit{adaptative worst-case adversary} qui génère le graphe à la
volée d'après l'évolution passée du graphe; l'\textit{oblivious
  worst-case adversary} impose un graphe pré-déterminé; et le
\textit{random-graph adversary}, lequel n'est pas vraiment un
adversaire mais une loi de probabilité déterminant l'évolution du
graphe.

\paragraph{Croissance de nœuds} Les auteurs définissent la \textit{vertex
  growth function} \(g : \mathbb{N} \to \mathbb{N}\), qui généralise
la notion de graphe expanseur. Un graphe (statique) \(G = (V, E)\) a
une croissance de nœuds \(g\) si pour tout ensemble de nœuds \(S
\subset V\) de taille \(s = |S| \leq |V|/2\), l'ensemble des voisins
\(N(S) := \{v \in V \setminus S : \exists u \in S : (u, v) \in E\}\)
est de taille au moins \(|N(S)| \geq g(|S|)\).

Par exemple, si \(G\) est connexe, sa croissance est au moins \(1\);
s'il est \(k\)-connexe, sa croissance est au moins \(g(s) = k\); si
\(G\) est un \(\alpha\)-expanseur, alors \(g(s) = \alpha s\).

On note \(g^{(f)}\) l'itérée après \(f\) tours, avec
\(g^{(0)}(s) = s\), et pour \(f > 0\),
\(g^{(f)}(s) = g^{(f-1)}(s) + g(g^{(f-1)}(s))\).

\textcolor{red}{FIXME:\@ clarifier, exemple}.\\

Si chaque graphe statique est un expanseur, le diamètre dynamique est
\(O(\log n)\); s'ils sont \(k\)-connexe, le diamètre est \(O(n/k)\).
\textcolor{red}{FIXME:\@ quote, preuve}.

\paragraph{Lemme} Soit \(G = (V, E)\) un graphe dynamique tel que pour
chaque tour \(r\), le graphe statique \(G(r)\) est \(g\)-croissant. Le
diamètre dynamique de \(G\) est au plus \(2d\), où \(d\) est le plus
petit entier tel que \(g^{(d)}(1) > |V|/2\).\\

Les auteurs considèrent ensuite le problème de la synchronisation
d'horloges dans un réseau. Cela peut être utile pour passer
discrétiser le temps, par exemple pour effectuer une action toutes les
\(t\) secondes.

Les auteurs prouvent les bornes inférieure et supérieure du
diamètre dynamique pour les graphes de Markov.\\

Afin de pouvoir calculer dans un graphe (par exemple compter le
nombres de nœuds, en supposant qu'il ne varie pas), les auteurs
explorent un problème intermédiaire, le \(k\)-comité.

\printbibliography{}

\end{document}
