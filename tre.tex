\documentclass[12pt,a4paper]{article}
\usepackage{polyglossia}
\setmainlanguage{french}
\usepackage{amsmath,amssymb,amsthm}
% \usepackage[margin=1.5cm]{geometry}
\usepackage{tikz}
\usetikzlibrary{positioning}
\usepackage{subfig}

\title{Graphes Temporels ou Dynamiques}
\author{Antonin Décimo\\
Encadrants~: Michel Habib, Laurent Viennot}

\begin{document}
\maketitle
\section{Introduction}
Un graphe dynamique est un graphe qui évolue au cours du temps,
c'est-à-dire que des nœuds ou des arêtes peuvent apparaître ou
disparaître au cours du temps. On peut s'en servir pour modéliser des
réseaux informatiques, comme du calcul distribué ou de la téléphonie
mobile, ou bien encore des réseaux de transports routiers ou publics.

Plusieurs modèles existent dans la littérature, un des plus simples
étant de considérer un ensemble de nœuds fixé, et seulement des arêtes
qui évoluent entre ces nœuds, ce qui amène à considérer une suite de
graphes statiques.

\begin{figure}[h]
  \centering
  \subfloat[\(G_1\)]{
    \begin{tikzpicture}[node/.style={circle, draw, minimum size=1cm]},
      scale=0.5, transform shape]
      \node[node] (e) {e};
      \node[node, above left=of e] (a) {a};
      \node[node, above right=of e] (b) {b};
      \node[node, below right=of e] (c) {c};
      \node[node, below left=of e] (d) {d};

      \path (a) edge (b);
      \path (e) edge (c);
    \end{tikzpicture}}
  \qquad
  \subfloat[\(G_2\)]{
    \begin{tikzpicture}[node/.style={circle, draw, minimum size=1cm]},
      scale=0.5, transform shape]
      \node[node] (e) {e};
      \node[node, above left=of e] (a) {a};
      \node[node, above right=of e] (b) {b};
      \node[node, below right=of e] (c) {c};
      \node[node, below left=of e] (d) {d};

      \path (a) edge (b) edge [bend left=30] (c);
    \end{tikzpicture}}
  \qquad
  \subfloat[\(G_3\)]{
    \begin{tikzpicture}[node/.style={circle, draw, minimum size=1cm]},
      scale=0.5, transform shape]
      \node[node] (e) {e};
      \node[node, above left=of e] (a) {a};
      \node[node, above right=of e] (b) {b};
      \node[node, below right=of e] (c) {c};
      \node[node, below left=of e] (d) {d};

      \path (c) edge (d) edge (e);
      \path (a) edge (e);
    \end{tikzpicture}}
  \qquad
  \subfloat[\(G_4\)]{
    \begin{tikzpicture}[node/.style={circle, draw, minimum size=1cm]},
      scale=0.5, transform shape]
      \node[node] (e) {e};
      \node[node, above left=of e] (a) {a};
      \node[node, above right=of e] (b) {b};
      \node[node, below right=of e] (c) {c};
      \node[node, below left=of e] (d) {d};

      \path (b) edge (e) edge (c);
      \path (c) edge (d);
    \end{tikzpicture}}
  \caption{Graphe dynamique \(G = (G_1, G_2, G_3, G_4)\)}
\end{figure}

\subsection{Rappels}
Rappelons d'abord quelques notions sur les graphes, qui nous seront
utiles par la suite.

\begin{description}
\item[Connectivité, connexité]
\item[\(k\)-connectivité]
\item[Graphe expanseur]
\item[Graphe \(d\)-régulier]
\end{description}

\section{Modèles de graphes dynamiques}

\newtheorem{lamport_causality}{Causalité de Lamport}
\begin{lamport_causality}
  Soit un graphe dynamique \(G = (V, E, T)\), on peut définir un ordre
  \(\rightarrow\) sur \((V, T)\), tel que
  \((u, r) \rightarrow (v, r')\) ssi \((r' = r + 1)\) et
  \(\{u, v\} \in E(r)\). L'\textit{ordre causal} \(\rightsquigarrow\)
  est défini comme clôture réflexive et transitive de \(\rightarrow\).
\end{lamport_causality}

\section{Problèmes algorithmiques}

Quels sont les problèmes algorithmiques que l'on peut rencontrer dans
ce cadre~?

\begin{description}
  \item[Diamètre dynamique] (\textit{dynamic diameter},
    \textit{flooding time}) Soit \(G = (V, E)\) un graphe dynamique et
    \(u \in V\). Si \(\forall v \in V, (u, t) \rightsquigarrow (v, t' + D)\),
    alors \(D\) est le diamètre dynamique du graphe. Il faut
    comprendre que si \(u\) envoie un message au temps \(t\), alors le
    message mettra au maximum \(D\) tours pour atteindre tous les
    nœuds du graphe.
  \item[Couverture] (\textit{cover time})
  \item[Plus court chemin] L'algorithme de Dijkstra fonctionne dans
    les graphes standards en remarquant que le préfixe d'un plus court
    chemin est aussi un plus court chemin. Or, cette propriété n'est
    pas vérifiée dans les graphes dynamiques.
\end{description}


\pagebreak

\section{Notions}

\paragraph{Transports Publics}
L'idée est de représenter un réseau de transport public (métros,
trams, bus, …) par un graphe, et d'étudier les modèles et les
algorithmes pour une panoplie de problèmes~:
\begin{itemize}
\item Recherche du chemin le plus rapide
\item Recherche du chemin avec la date d'arrivée la plus précoce
\item Multi-critères~: avec le moins de changements
\item Rechercher tous les chemins dans une plage horaire
\end{itemize}


\paragraph{Graphes Dynamiques}
Les modèles varient autour de l'idée de \textit{graphes
  dynamiques}. Les noeuds représentent des stations, et les arrêtes
les lignes entre les stations.

\begin{itemize}
\item Un graphe dynamique peut être vu comme une séquence de graphes
  au cours du temps \(\mathcal{N} = \dots, \mathcal{N}_{t-1},
  \mathcal{N}_{t}, \mathcal{N}_{t+1}, \dots\). La latence d'un arc ou
  d'un noeud est une simple étiquette du noeud~\cite{xuan2003computing}.
\item Une autre présentation~\cite{casteigts2012time}~:
  \(\mathcal{G} = (V, E, \mathcal{T}, \rho, \zeta, \psi, \varphi)\),
  où~:
  \begin{itemize}
  \item \(\rho : E \times \mathcal{T} \to \mathbb{B}\), fonction de
    présence des arrêtes,
  \item \(\zeta : E \times \mathcal{T} \to \mathbb{T}\), fonction de
    latence des arrêtes,
  \item \(\psi : V \times \mathcal{T} \to \mathbb{B}\), fonction de
    présence des noeuds,
  \item \(\varphi : V \times \mathcal{T} \to \mathbb{T}\), fonction de
    latence des noeuds
  \end{itemize}
\end{itemize}

\paragraph{Points de vue}
Dans~\cite{casteigts2012time}, on parle de point de vue de
l'évolution. Du point de vue de l'arrête, sa disponibilité et sa
latence change au cours du temps. Pour un noeud, c'est son voisinage.
Pour tout le graphe, c'est un événement. Chaque événement définit un
nouveau sous-graphe statique. Le graphe dynamique est une séquence de
graphes statiques tels que \(G_i \neq G_{i+1}\).

\paragraph{Sous-graphes}
On définit assez naturellement la notion de sous-graphe et de
\textit{sous-graphe temporel} dans~\cite{casteigts2012time}
et~\cite{latapy2017stream}.

\paragraph{Voyage}
Un voyage est un chemin dans le graphe respectant les contraintes
temporelles.

\paragraph{Distance}
Deux notions de distance (sur un voyage)~:
\begin{itemize}
\item le voyage le plus court (\textit{hop-count})
\item le voyage le plus rapide (qui prends le moins de temps, ou qui
  termine le premier) (\textbf{toujours pas clair})
\end{itemize}

\paragraph{Algorithmes}
Est introduite en~\cite{xuan2003computing} deux notions de graphes
dynamiques, les \textit{evolving graphs (EG)} et les \textit{timed
  evolving graphs (TEG)}, la différence étant que les \textit{TEG}
disposent d'une fonction de latence sur les arrêtes.

Pour appliquer l'algorithme de Dijkstra, il faut remarquer que dans un
graphe normal, le préfixe d'un plus court chemin est aussi un plus
court chemin. Or, cette propriété n'est pas vérifiée dans les graphes
dynamiques. Il faut donc trouver à chaque fois un chemin préfixe qui
respecte cette propriété.

On peut alors calculer les plus courts (\textit{hop-count}) chemins,
et (plus intéressant) les chemins les plus rapides pour toutes les
paires de chemins.

\section{Papiers}

\paragraph{Time-varying graphs and dynamic networks}
Introduction du concept de TVG.\@Classification des TVG en fonction
des propriétés de connexion du graphe. Etude des phénomènes tels que
le \textit{petit monde}, \textit{fairness \& balance}. Graphes
dynamiques où la fonction de présence suit une loi de probabilité.

\paragraph{Stream graphs and link streams for the modeling of
  interactions over time}
Gros papier qui étend toutes les relations et les sous-structures sur
les graphes aux graphes dynamiques.

\bibliographystyle{plain}
\bibliography{tre}

\end{document}
